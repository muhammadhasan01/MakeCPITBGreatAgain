\documentclass{article}

\usepackage{geometry}
\usepackage{amsmath}
\usepackage{graphicx}
\usepackage{listings}
\usepackage{hyperref}
\usepackage{multicol}
\usepackage{fancyhdr}
\pagestyle{fancy}
\hypersetup{ colorlinks=true, linkcolor=black, filecolor=magenta, urlcolor=cyan}
\geometry{ a4paper, total={170mm,257mm}, top=20mm, right=20mm, bottom=20mm, left=20mm}
\setlength{\parindent}{0pt}
\setlength{\parskip}{1em}
\renewcommand{\headrulewidth}{0pt}
\lhead{Competitive Programming - ITB}
\fancyfoot[CE,CO]{\thepage}
\lstset{
    basicstyle=\ttfamily\small,
    columns=fixed,
    extendedchars=true,
    breaklines=true,
    tabsize=2,
    prebreak=\raisebox{0ex}[0ex][0ex]{\ensuremath{\hookleftarrow}},
    frame=none,
    showtabs=false,
    showspaces=false,
    showstringspaces=false,
    prebreak={},
    keywordstyle=\color[rgb]{0.627,0.126,0.941},
    commentstyle=\color[rgb]{0.133,0.545,0.133},
    stringstyle=\color[rgb]{01,0,0},
    captionpos=t,
    escapeinside={(\%}{\%)}
}

\begin{document}

\begin{center}
    \section*{Naik Tangga (Version 1)} % ganti judul soal

    \begin{tabular}{ | c c | }
        \hline
        Batas Waktu  & 1s \\    % jangan lupa ganti time limit
        Batas Memori & 512MB \\  % jangan lupa ganti memory limit
        \hline
    \end{tabular}
\end{center}

\subsection*{Deskripsi}

Hojun memiliki rumah yang sangat mewah, sehingga rumah ini sangat megah dan tinggi. Didalam rumah tersebut terdapat $N$ tangga. Hojun biasanya menaiki tangga tersebut dengan dua cara:
\vspace{-\baselineskip}
\begin{enumerate}
    \setlength\itemsep{0pt}
    \item Naik hanya satu tangga
    \item Naik langsung dua tangga
\end{enumerate}
\vspace{-\baselineskip}
Hojun sadar ternyata ada banyak cara menaiki tangga tersebut. Hojun-pun memiliki $Q$ pertanyaan, di setiap pertanyaan tersebut Hojun ingin tahu ada berapa banyak cara menaiki tangga ke-$x$ dari $N$ tangga tersebut. Bantulah Hojun menyelesaikan permasalahan ini!

\subsection*{Format Masukan}

Baris pertama terdiri dari dua bilangan $N$ dan $Q$ ($1 \leq N, Q \leq 100000$) menyatakan banyaknya tangga dan banyaknya pertanyaan yang hojun miliki

Baris $Q$ selanjutnya berisi satu bilangan $x$ ($1 \leq x \leq N$) menyatakan pertanyaan mengenai banyaknya cara menaiki tangga ke-$x$ tersebut.

\subsection*{Format Keluaran}

Keluarkan $Q$ baris dengan baris ke-$i$ merupakan jawaban mengenai banyaknya cara menaiki tangga pertanyaan ke-$i$. Karena jawaban ini bisa sangat besar, keluarkan jawaban tersebut dalam modulo $10^9 + 7$
\\

\begin{multicols}{2}
\subsection*{Contoh Masukan}
\begin{lstlisting}
5 3
1
2
3
\end{lstlisting}
\columnbreak
\subsection*{Contoh Keluaran}
\begin{lstlisting}
1
2
3
\end{lstlisting}
\vfill
\null
\end{multicols}

\subsection*{Penjelasan}
Untuk $x = 3$, kita bisa mendapat tiga kemungkinan yaitu
\vspace{-\baselineskip}
\begin{enumerate}
    \setlength\itemsep{0pt}
    \item Naik satu tangga sebanyak tiga kali
    \item Naik satu tangga kemudian Naik langsung dua tangga
    \item Naik langsung dua tangga kemudian Naik satu tangga
\end{enumerate}

\pagebreak

\end{document}