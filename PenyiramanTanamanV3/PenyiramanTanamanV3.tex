\documentclass{article}

\usepackage{geometry}
\usepackage{amsmath}
\usepackage{graphicx}
\usepackage{listings}
\usepackage{hyperref}
\usepackage{multicol}
\usepackage{fancyhdr}
\pagestyle{fancy}
\hypersetup{ colorlinks=true, linkcolor=black, filecolor=magenta, urlcolor=cyan}
\geometry{ a4paper, total={170mm,257mm}, top=20mm, right=20mm, bottom=20mm, left=20mm}
\setlength{\parindent}{0pt}
\setlength{\parskip}{1em}
\renewcommand{\headrulewidth}{0pt}
\lhead{Competitive Programming - ITB}
\fancyfoot[CE,CO]{\thepage}
\lstset{
    basicstyle=\ttfamily\small,
    columns=fixed,
    extendedchars=true,
    breaklines=true,
    tabsize=2,
    prebreak=\raisebox{0ex}[0ex][0ex]{\ensuremath{\hookleftarrow}},
    frame=none,
    showtabs=false,
    showspaces=false,
    showstringspaces=false,
    prebreak={},
    keywordstyle=\color[rgb]{0.627,0.126,0.941},
    commentstyle=\color[rgb]{0.133,0.545,0.133},
    stringstyle=\color[rgb]{01,0,0},
    captionpos=t,
    escapeinside={(\%}{\%)}
}

\begin{document}

\begin{center}
    \section*{Penyiraman Tanaman (Version 3)} % ganti judul soal

    \begin{tabular}{ | c c | }
        \hline
        Batas Waktu  & 2s \\    % jangan lupa ganti time limit
        Batas Memori & 512MB \\  % jangan lupa ganti memory limit
        \hline
    \end{tabular}
\end{center}

\subsection*{Deskripsi}
Disaat kita disuruh \#DirumahSaja, Mufraswid kini menjadi giat menyiram tanaman yang ada di halaman rumahnya. Terdapat $N$ tanaman yang dia miliki, pada awalnya tanaman ke-$i$ ($1 \leq i \leq N$) memiliki tingkat kesegaran tanaman sebesar $T_i$. Setiap kali Mufraswid menyirami suatu tanaman, maka kesegaran-nya akan bertambah satu. Mufraswid bosan hanya menyiram satu tanaman saja, sekarang dia memiliki alat baru yang dapat menyiram tanaman dari suatu rentang $\left[L, R \right]$ ($1 \leq L \leq R \leq N$), sehingga setiap tanaman pada rentang tersebut akan bertambah kesegaran tanamannya sebanyak satu, dan pada saat itu, seperti biasa, terkadang dia suka menanyakan kepada dirinya, untuk suatu rentang $\left[L, R \right]$ ($1 \leq L \leq R \leq N$) berapa maksimal kesiraman tanaman dan jumlah kesegaran tanaman dari rentang tersebut. Kebetulan Mufraswid meminta Anda membuat program untuk menjawab pertanyaan rentang tersebut! Bantulah Mufraswid ini!


\subsection*{Format Masukan}

Baris pertama terdiri dari satu bilangan bulat positif $N$ ($1 \leq N \leq 100000$).

Baris selanjutnya berisi $N$ bilangan $T_1, T_2, ..., T_N$ ($0 \leq T_i \leq 100000$) yang menyatakan kesegaran setiap tanaman pada awalnya.

Baris selanjutnya terdiri dari satu bilangan $Q$ ($1\leq Q\leq 100000$) yang menyatakan banyaknya kegiatan Mufraswid saat menyiram tanaman-nya.

Kemudian $Q$ baris selanjutnya berisi salah satu dari dua kemungkinan kegiatan:
\vspace{-\baselineskip}
\begin{itemize}
    \setlength\itemsep{0pt}
	\item 1 $L$ $R$ (Siram dari tanaman ke-$L$ sampai tanaman ke-$R$, $1 \leq L \leq R \leq N$)
	\item 2 $L$ $R$ (Pertanyaan mengenai maksimal dan jumlah kesegaran tanaman dari $L$ sampai $R$, $1 \leq L \leq R \leq N$)
\end{itemize}
\vspace{-\baselineskip}
(Dipastikan terdapat minimal satu buah pertanyaan dari $Q$ kegiatan tersebut)

\subsection*{Format Keluaran}

Untuk setiap pertanyaan, keluarkan jawaban dalam satu baris berisi dua bilangan yang merupakan maksimal kesegaran tanaman dan jumlah kesegaran tanaman pada rentang yang bersangkutan.

\begin{multicols}{2}
\subsection*{Contoh Masukan}
\begin{lstlisting}
5
1 3 5 0 1
6
2 1 4
1 1 5
2 1 4
1 3 5
2 3 5
2 3 3
\end{lstlisting}
\columnbreak
\subsection*{Contoh Keluaran}
\begin{lstlisting}
5 9
6 13
7 12
7 7

\end{lstlisting}
\vfill
\null
\end{multicols}

% \subsection*{Penjelasan}
% Jika dibutuhkan, tambahkan penjelasan di sini

\pagebreak

\end{document}