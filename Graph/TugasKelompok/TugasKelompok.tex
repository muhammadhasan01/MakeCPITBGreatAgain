\documentclass{article}

\usepackage{geometry}
\usepackage{amsmath}
\usepackage{graphicx}
\usepackage{listings}
\usepackage{hyperref}
\usepackage{multicol}
\usepackage{fancyhdr}
\pagestyle{fancy}
\hypersetup{ colorlinks=true, linkcolor=black, filecolor=magenta, urlcolor=cyan}
\geometry{ a4paper, total={170mm,257mm}, top=20mm, right=20mm, bottom=20mm, left=20mm}
\setlength{\parindent}{0pt}
\setlength{\parskip}{1em}
\renewcommand{\headrulewidth}{0pt}
\lhead{Competitive Programming - ITB}
\fancyfoot[CE,CO]{\thepage}
\lstset{
    basicstyle=\ttfamily\small,
    columns=fixed,
    extendedchars=true,
    breaklines=true,
    tabsize=2,
    prebreak=\raisebox{0ex}[0ex][0ex]{\ensuremath{\hookleftarrow}},
    frame=none,
    showtabs=false,
    showspaces=false,
    showstringspaces=false,
    prebreak={},
    keywordstyle=\color[rgb]{0.627,0.126,0.941},
    commentstyle=\color[rgb]{0.133,0.545,0.133},
    stringstyle=\color[rgb]{01,0,0},
    captionpos=t,
    escapeinside={(\%}{\%)}
}

\begin{document}

\begin{center}
    \section*{Tugas Kelompok} % ganti judul soal

    \begin{tabular}{ | c c | }
        \hline
        Batas Waktu  & 1s \\    % jangan lupa ganti time limit
        Batas Memori & 512MB \\  % jangan lupa ganti memory limit
        \hline
    \end{tabular}
\end{center}

\subsection*{Deskripsi}
Sebuah mata kuliah mengharuskan mahasiswa membuat kelompok untuk tugas kelompok. Karena dosen tidak ingin membuatkan kelompok, dia membebaskan mahasiswa untuk membentuk kelompok dengan syarat berikut:

1. Banyak kelompok yang terbentuk harus seminimal mungkin,
\newline
2. Sebuah anggota kelompok harus memiliki minimal 1 teman yang berada di satu kelompok yang sama,
\newline
3. Satu kelompok terdiri dari minimal 1 orang.

Diketahui bahwa pada kelas tersebut terdapat $N$ mahasiswa dan $M$ pertemanan.

Sebuah pertemanan $F_i$ dilambangkan dengan angka $A_i$ dan $B_i$ yang berarti $A_i$ berteman dengan $B_i$ dan $B_i$ berteman dengan $A_i$.

Anda diminta untuk membantu mereka membentuk kelompok. Tentukan berapa banyak kelompok yang akan terbentuk.

\subsection*{Format Masukan}

Baris pertama terdiri dari dua bilangan bulat positif $N$, $M$ dipisahkan satu spasi yang menyatakan banyak anak pada satu kelas dan banyak pertemanan ($1 \leq N, M \leq 200000$).

$M$ baris berikutnya berisi masing-masing dua buah bilangan $A_i$ dan $B_i$ yang menyatakan hubungan pertemanan antara $A_i$ dan $B_i$ ($1 \leq A_i, B_i \leq N$), ($(A_i, B_i) \neq (A_j, B_j)$ dan $(A_i, B_i) \neq (B_j, A_j)$ dengan ($i \neq j$)), ($1 \leq i, j \leq M$).


\subsection*{Format Keluaran}

Keluarkan satu baris yang berisi bilangan yang menyatakan banyak kelompok yang terbentuk.

\newline
\begin{multicols}{2}
\subsection*{Contoh Masukan}
\begin{lstlisting}
5 5
1 2
2 3
3 4
3 5
1 5
\end{lstlisting}
\columnbreak
\subsection*{Contoh Keluaran}
\begin{lstlisting}
1
\end{lstlisting}
\vfill
\null
\end{multicols}

\begin{multicols}{2}
\subsection*{Contoh Masukan}
\begin{lstlisting}
6 3
1 2
2 6
3 4
\end{lstlisting}
\columnbreak
\subsection*{Contoh Keluaran}
\begin{lstlisting}
3
\end{lstlisting}
\vfill
\null
\end{multicols}

% \subsection*{Penjelasan}
% Jika dibutuhkan, tambahkan penjelasan di sini

\pagebreak

\end{document}