\documentclass{article}

\usepackage{geometry}
\usepackage{amsmath}
\usepackage{graphicx}
\usepackage{listings}
\usepackage{hyperref}
\usepackage{multicol}
\usepackage{fancyhdr}
\pagestyle{fancy}
\hypersetup{ colorlinks=true, linkcolor=black, filecolor=magenta, urlcolor=cyan}
\geometry{ a4paper, total={170mm,257mm}, top=20mm, right=20mm, bottom=20mm, left=20mm}
\setlength{\parindent}{0pt}
\setlength{\parskip}{1em}
\renewcommand{\headrulewidth}{0pt}
\lhead{Competitive Programming - ITB}
\fancyfoot[CE,CO]{\thepage}
\lstset{
    basicstyle=\ttfamily\small,
    columns=fixed,
    extendedchars=true,
    breaklines=true,
    tabsize=2,
    prebreak=\raisebox{0ex}[0ex][0ex]{\ensuremath{\hookleftarrow}},
    frame=none,
    showtabs=false,
    showspaces=false,
    showstringspaces=false,
    prebreak={},
    keywordstyle=\color[rgb]{0.627,0.126,0.941},
    commentstyle=\color[rgb]{0.133,0.545,0.133},
    stringstyle=\color[rgb]{01,0,0},
    captionpos=t,
    escapeinside={(\%}{\%)}
}

\begin{document}

\begin{center}
    \section*{Potong Kue} % ganti judul soal

    \begin{tabular}{ | c c | }
        \hline
        Batas Waktu  & 2s \\    % jangan lupa ganti time limit
        Batas Memori & 512MB \\  % jangan lupa ganti memory limit
        \hline
    \end{tabular}
\end{center}

\subsection*{Deskripsi}
Hanas dan Sarraf mendapat sebuah kue yang memiliki $N$ baris dan $M$ kolom. Kue ini memiliki $K$ topping yang berada pada posisi tertentu. Topping ke $i$ kue berada di posisi baris $R_i$ dan kolom $C_i$.

Karena hanas dan sarraf sangat menyukai topping kue, mereka ingin membagi
kue ini menjadi dua bagian sehingga jumlah topping pada bagian satu dan 
yang lain memiliki perbedaan seminim mungkin.

Pembagian kue akan dilakukan dengan memotong kue secara vertikal pada jarak $D$ dari kiri kue. (potongan membentuk garis lurus vertikal)

Carilah berapa beda terkecil dari banyak topping pada kedua bagian kue.

\subsection*{Format Masukan}

Baris pertama terdiri dari tiga bilangan bulat positif $N$, $M$, dan $K$ dipisahkan satu spasi yang menyatakan tinggi kue, lebar kue, dan banyak topping ($1 \leq N \leq 10$), ($2 \leq M \leq 100000$), ($1 \leq K \leq 200000$).

$K$ baris berikutnya berisi masing-masing dua buah bilangan $R_i$ dan $C_i$ yang menyatakan posisi topping ke-$i$ pada kue. ($1 \leq R_i \leq N$), ($1 \leq C_i \leq M$), ($1 \leq i \leq K$).

\subsection*{Format Keluaran}

Keluarkan satu baris yang berisi bilangan yang menyatakan beda terkecil yang bisa didapatkan dari topping di kedua bagian hasil potongan. (perhatikan bahwa masing-masing topping harus berada pada salah satu bagian)

\newline
\begin{multicols}{2}
\subsection*{Contoh Masukan}
\begin{lstlisting}
1 4 4
1 2
1 2
1 1
1 1
\end{lstlisting}
\columnbreak
\subsection*{Contoh Keluaran}
\begin{lstlisting}
0
\end{lstlisting}
\vfill
\null
\end{multicols}

\begin{multicols}{2}
\subsection*{Contoh Masukan}
\begin{lstlisting}
2 4 5
1 1
1 3
2 3
2 3
1 4
\end{lstlisting}
\columnbreak
\subsection*{Contoh Keluaran}
\begin{lstlisting}
3
\end{lstlisting}
\vfill
\null
\end{multicols}

% \subsection*{Penjelasan}
% Jika dibutuhkan, tambahkan penjelasan di sini

\pagebreak

\end{document}