\documentclass{article}

\usepackage{geometry}
\usepackage{amsmath}
\usepackage{graphicx}
\usepackage{listings}
\usepackage{hyperref}
\usepackage{multicol}
\usepackage{fancyhdr}
\pagestyle{fancy}
\hypersetup{ colorlinks=true, linkcolor=black, filecolor=magenta, urlcolor=cyan}
\geometry{ a4paper, total={170mm,257mm}, top=20mm, right=20mm, bottom=20mm, left=20mm}
\setlength{\parindent}{0pt}
\setlength{\parskip}{1em}
\renewcommand{\headrulewidth}{0pt}
\lhead{Competitive Programming - ITB}
\fancyfoot[CE,CO]{\thepage}
\lstset{
    basicstyle=\ttfamily\small,
    columns=fixed,
    extendedchars=true,
    breaklines=true,
    tabsize=2,
    prebreak=\raisebox{0ex}[0ex][0ex]{\ensuremath{\hookleftarrow}},
    frame=none,
    showtabs=false,
    showspaces=false,
    showstringspaces=false,
    prebreak={},
    keywordstyle=\color[rgb]{0.627,0.126,0.941},
    commentstyle=\color[rgb]{0.133,0.545,0.133},
    stringstyle=\color[rgb]{01,0,0},
    captionpos=t,
    escapeinside={(\%}{\%)}
}

\begin{document}

\begin{center}
    \section*{Tugas Besar Mufraswid} % ganti judul soal

    \begin{tabular}{ | c c | }
        \hline
        Batas Waktu  & 1s \\    % jangan lupa ganti time limit
        Batas Memori & 512MB \\  % jangan lupa ganti memory limit
        \hline
    \end{tabular}
\end{center}

\subsection*{Deskripsi}

Setelah "mengerjakan" tugas kecilnya, tiba - tiba muncul notifikasi email dari seorang asisten: "[IFXXXX] TUGAS BESAR 1". Mufraswid panik. Ada 10 tugas besar lainnya yang belum ia selesaikan. Ia pun pergi ke C*kapundung untuk mencari bantuan. Kamu, sebagai seorang mahasiswa yang sedang magang di C*kapundung pun diminta untuk mengerjakan tugasnya.
\par
Kamu diberikan sebuah array yang berukuran $R \times C$. Kamu diminta untuk mencari nilai terbesar dari suatu subrectangle (yang tentunya tidak boleh kosong). Nilai dari subrectangle adalah jumlah dari setiap elemennya.
\par
Subrectangle adalah bagian dari sebuah array 2 dimensi yang membentuk suatu persegi/persegi panjang. Untuk lebih jelasnya, lihat contoh dibawah.

\subsection*{Format Masukan}

Baris pertama terdiri dari 2 buah bilangan bulat positif $R$ dan $C$, yaitu banyaknya baris dan kolom.
\par
$R$ baris berikutnya terdiri dari $C$ buah bilangan, $A_{ij}$, yaitu elemen array.

\subsection*{Format Keluaran}

1 buah bilangan, yaitu nilai terbesar yang mungkin didapat dari suatu subrectangle.

\subsection*{Batasan}

\begin{itemize}
    \setlength\itemsep{0pt}
    \item $1 \leq R, C \leq 100$
    \item $-10^9 \leq A_{ij} \leq 10^9$ ($1 \leq i \leq R, 1 \leq j \leq C$)
\end{itemize}
\begin{multicols}{2}
\subsection*{Contoh Masukan}
\begin{lstlisting}
3 2
1 3
-4 6
2 -3
\end{lstlisting}
\columnbreak
\subsection*{Contoh Keluaran}
\begin{lstlisting}
9
\end{lstlisting}
\vfill
\null
\end{multicols}

\begin{multicols}{2}
\subsection*{Contoh Masukan}
\begin{lstlisting}
1 5
-5 2 3 -2 4
\end{lstlisting}
\columnbreak
\subsection*{Contoh Keluaran}
\begin{lstlisting}
7
\end{lstlisting}
\vfill
\null
\end{multicols}
\subsection*{Penjelasan}
Pada contoh pertama:
\newline
1 \textbf{3}
\newline
-4 \textbf{6}
\newline
2 -3
\newline
Nilainya adalah $3+6 = 9$.

\pagebreak

\end{document}